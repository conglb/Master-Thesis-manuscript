Advances in shipbuilding, engine technology, and navigation over the centuries have significantly expanded global trade, making maritime transport indispensable. In recent decades, digitalization has emerged as a key development in marine shipping, reshaping the maritime industry to meet the growing volume of goods, the need for cost-efficient operations, and the demand for reducing environmental impact. Today, many decisions in the maritime industry are driven by data analytics from on-board systems, highlighting the value of ship digitization. For example, Maersk Line has installed IoT sensors on its 600 vessels to enhance fuel economy, optimize voyages, and monitor and optimize container conditions.
The intelligent use of operational data from vessels plays a crucial role in transforming the future of maritime services. The maritime sector generates vast amounts of data from a variety of sources, including sensors, cameras, radar, GPS, and communication technologies such as Wi-Fi and Bluetooth. As a result, the volume of global ship data has grown exponentially, reaching the scale of "big data," but with limited standardization and actionable insights. When operational vessel data is effectively managed, on-board data could allow any ship to serve as a mobile observation platform, providing valuable information for research in fields such as shipbuilding, meteorology, and more. For instance, AIS data, a global standard for vessel tracking, is mandatory for ships over 2 tons. MarineTraffic.com, for example, collected around 30 GB of AIS data daily in 2013, and this figure has been steadily increasing.
Despite the potential of such data, the maritime industry faces a significant challenge: the lack of universally shared standards for operational data across the marine transportation ecosystem, such as between different fleet management systems. This issue is partly due to concerns about data privacy and the lack of interoperability between the various IT systems involved in maritime logistics. Without a transparent data-processing framework shared among shipowners, integrating operational vessel data requires substantial time and effort for practitioners and researchers engaged in data science tasks.
To address this challenge, this thesis introduces a data-processing framework for operational vessel data that can assimilate information from various data owners. The proposed framework is designed to automatically acquire, clean, and consolidate data from heterogeneous sources (e.g., files, XML, JSON, databases) located in different locations. It ensures that data from various formats and locations can be stored in a common database, enabling a wide range of analyses. The data pipeline in the framework is also expected to handle large volumes of operational vessel data (OVD) semi-automatically from numerous vessels. Additionally, the framework is designed to be easy to understand, as it is based on well-established data science workflows.

==============